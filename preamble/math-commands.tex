%%%%%%%%%%%%%%%%%%%%%%%%%%%%%%%%%%%%%
%%%   Custom math mode commands   %%%
%%%%%%%%%%%%%%%%%%%%%%%%%%%%%%%%%%%%%
%
%
% Modify styles for theorems and proofs
%\numberwithin{equation}{subsection}
%\renewcommand{\proofname}{Beweis}
%\theoremstyle{varplain}
%\newtheorem{theorem}{Satz}[subsection]
%\newtheorem{lemma}[theorem]{Lemma}
%\newtheorem{corollary}[theorem]{Korollar}
%\newtheorem{remark}[theorem]{Bemerkung}
%\theoremstyle{vardefinition}
%\newtheorem{definition}[theorem]{Definition}
%\newtheorem{example}[theorem]{Beispiel}

% Define sans serif math fonts
\DeclareMathVersion{sans}
\SetSymbolFont{operators}{sans}{OT1}{cmbr}{m}{n}
\SetSymbolFont{letters}{sans}{OML}{cmbrm}{m}{it}
\SetSymbolFont{symbols}{sans}{OMS}{cmbrs}{m}{n}
\SetMathAlphabet{\mathit}{sans}{OT1}{cmbr}{m}{sl}
\SetMathAlphabet{\mathbf}{sans}{OT1}{cmbr}{bx}{n}
\SetMathAlphabet{\mathtt}{sans}{OT1}{cmtl}{m}{n}
\SetSymbolFont{largesymbols}{sans}{OMX}{iwona}{m}{n}

% Define symbols for sets
\usepackage{type1cm}
\makeatletter
\newcommand*{\getfontsize}{\f@size pt}
\makeatother
\newfont{\tinydsrom}{dsrom10 scaled 500}
\newfont{\scriptsizedsrom}{dsrom10 scaled 700}
\newfont{\footnotesizedsrom}{dsrom10 scaled 800}
\newfont{\smalldsrom}{dsrom10 scaled 900}
\newfont{\normalsizedsrom}{dsrom10 scaled 1000}
\newfont{\largedsrom}{dsrom10 scaled 1200}
\newfont{\Largedsrom}{dsrom10 scaled 1440}
\newfont{\LARGEdsrom}{dsrom10 scaled 1728}
\newfont{\hugedsrom}{dsrom10 scaled 2074}
\newfont{\Hugedsrom}{dsrom10 scaled 2488}
\newlength{\dsromsize}
\newcommand*{\sizeddsrom}[1]{\setlength{\dsromsize}{\getfontsize}%
  \ifdim\dsromsize=5.0pt\tinydsrom\else%
  \ifdim\dsromsize=7.0pt\scriptsizedsrom\else%
  \ifdim\dsromsize=8.0pt\footnotesizedsrom\else%
  \ifdim\dsromsize=9.0pt\smalldsrom\else%
  \ifdim\dsromsize=12.0pt\largedsrom\else%
  \ifdim\dsromsize=14.4pt\Largedsrom\else%
  \ifdim\dsromsize=17.28pt\LARGEdsrom\else%
  \ifdim\dsromsize=20.74pt\hugedsrom\else%
  \ifdim\dsromsize=24.88pt\Hugedsrom\else\normalsizedsrom\fi\fi\fi\fi\fi\fi\fi\fi\fi#1}
\newcommand{\fragileset}[1]{{\ifmmode\text{\sizeddsrom{#1}}\else\sizeddsrom{#1}\fi}}
\newcommand{\set}[1]{\protect\fragileset{#1}}

% Bold symbol with bold italic greek letters
\makeatletter
\newcommand*\boldsymbol@itgreek[1]{\bm{\mathbf{#1}}}
\makeatother

% Bold symbol with bold upright greek letters
% Code from tex.stackexchange.com/questions/167172
% Explanation of expl3 programming language in document
% "The LaTeX3 interfaces" available on CTAN
\makeatletter
\ExplSyntaxOn
\NewDocumentCommand\boldsymbol@upgreek{m}
 {
  \boldupgreek_vector:n { #1 }
 }
% Define command \boldupgreek_vector:n
\cs_new_protected:Npn \boldupgreek_vector:n #1
 {
  % Iterate \bolduprgreek_vector_inner:n over
  %all tokens in token list #1
  \tl_map_inline:nn { #1 }
   {
    \boldupgreek_vector_inner:n { ##1 }
   }
 }
% Define command \boldupgreek_vector_inner:n
\cs_new_protected:Npn \boldupgreek_vector_inner:n #1
 {
  \tl_if_in:VnTF \g_boldupgreek_latin_tl { #1 }
   {% we check whether the argument is in Latin letter token list
    \mathbf { #1 } % a Latin letter
   }
   {% if not a Latin letter, we check if it's in uppercase Greek
    % letter token list
    \tl_if_in:VnTF \g_boldupgreek_ucgreek_tl { #1 }
     {
      \bm { #1 } % a Greek uppercase letter
     }
     {% if not, we check if it's in lowercase Greek letter token list
      \tl_if_in:VnTF \g_boldupgreek_lcgreek_tl { #1 }
       {
        \boldupgreek_makeboldupright:n { #1 }
       }
       {% none of the above, just issue #1
        #1 % fall back
       }
     }
   }
 }
% Define command to replace command for greek letter by
% (bold) command for upright greek letter
\cs_new_protected:Npn \boldupgreek_makeboldupright:n #1
 {
  \bm { \use:c { up \cs_to_str:N #1 } }
 }
% Define and set token lists for latin, uppercase greek
% and lowercase greek letters.
\tl_new:N \g_boldupgreek_latin_tl
\tl_new:N \g_boldupgreek_ucgreek_tl
\tl_new:N \g_boldupgreek_lcgreek_tl
\tl_gset:Nn \g_boldupgreek_latin_tl
 {
  ABCDEFGHIJKLMNOPQRSTUVWXYZ
  abcdefghijklmnopqrstuvwxyz
 }
\tl_gset:Nn \g_boldupgreek_ucgreek_tl
 {
  \Gamma\Delta\Theta\Lambda\Pi\Sigma\Upsilon\Phi\Chi\Psi\Omega
 }
\tl_gset:Nn \g_boldupgreek_lcgreek_tl
 {
  \alpha\beta\gamma\delta\epsilon\zeta\eta\theta\iota\kappa
  \lambda\mu\nu\xi\pi\rho\sigma\tau\upsilon\phi\chi\psi\omega
  \varepsilon\vartheta\varpi\varphi\varsigma\varrho
 }
\ExplSyntaxOff
\makeatother

% User defined commands
\allowdisplaybreaks[2]
\DeclareMathOperator\re{Re}
\DeclareMathOperator\im{Im}
\DeclareMathOperator\sign{sign}
\newcommand*\phantomequals{{\ifmmode\text{\phantom{$\mbox{}=\mbox{}$}}\else\phantom{$\mbox{}=\mbox{}$}\fi}}
\newcommand*\phantomrightarrow{{\ifmmode\text{\phantom{$\mbox{}\rightarrow\mbox{}$}}\else\phantom{$\mbox{}\rightarrow\mbox{}$}\fi}}
\makeatletter
\let\vect\boldsymbol@itgreek
\makeatother
\newcommand*\hvect[1]{\hat{\vect{#1}}}